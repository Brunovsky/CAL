\documentclass[relatorio.tex]{subfiles}

\begin{document}
\section{Interface}
\label{sec:impl}

No seguimento das aulas de pesquisa aproximada e exata foram implementados algoritmos de pesquisa exata, algoritmos de distância de edição e os equivalentes algoritmos de pesquisa aproximada (\textit{fuzzy search}).

O utilizador do programa seleciona o algoritmo a utilizar

\begin{enumerate}[listparindent=2em]
\item {\bfseries Seleção do algoritmo.}

O utilizador seleciona o algoritmo a utilizar da lista.

\item {\bfseries Seleção da rua.}

O utilizador indica o nome da \emph{rua} a pesquisar com a sua posição. Dependendo do tipo de algoritmo, claro, o nome da rua encontrada deverá ser exatamente igual ou aproximadamente igual ao dado.

\item {\bfseries Seleção de uma das ruas encontradas.}

Se alguma rua for encontrada o utilizador deverá escolher aquela em que se encontra de entre as selecionadas. As ruas aparecem destacadas visualmente no grafo.

\item {\bfseries Seleção da rua de destino.}

Novamente o utilizador introduz o nome da rua para onde pretende evacuar, o sistema pesquisa o nome da rua com o mesmo algoritmo e pede ao utilizador novamente para escolher.

\item {\bfseries Melhor caminho.}

Com as duas ruas o sistema encontra o caminho mais rápido entre as duas ruas.
\end{enumerate}
\end{document}