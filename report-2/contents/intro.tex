\documentclass[relatorio.tex]{subfiles}

\newcommand*{\rarrow}[1]{\ensuremath{\overrightarrow{#1}}}
\newcommand*{\id}{\sf{id} }
\newcommand*{\pre}[1]{}

\newcolumntype{b}{>{\hsize=1.44\hsize}X}
\newcolumntype{n}{X}
\newcolumntype{s}{>{\hsize=.56\hsize}X}

\begin{document}
\section{Introdução}
\label{sec:intro}

\subsection{Definições preliminares}
\label{subsec:preliminaries}

Começamos com as seguintes definições:

\begin{definitions}
\item $P$ refere-se ao padrão a pesquisar, ou ao seu comprimento;

\item $T$ refere-se ao texto a ser pesquisado, ou ao seu comprimento.

\item $L$ é o alfabeto de $T$ e de $P$, ou a sua cardinalidade;

\item $E$ é o subconjunto de $L$ definido por $P$, ou a sua cardinalidade;

\item $A$ o texto a ser transformado num algoritmo de distância de edição;

\item $B$ o texto alvo num tal algoritmo.
\end{definitions}



\subsection{Tema 3}
\label{subsec:tema}

\subsubsection{Especificação do Tema}
\label{subsubsec:spec}

A especificação e descrição do Tema 3 -- Sistema de Evacuação
é a seguinte:

\begin{quote}
Para a segunda parte deste trabalho, considere que as ruas têm nomes, por exemplo “Rua de Dr Roberto Frias” ou “A1”, e que pertencem a um dado distrito, por exemplo, “Porto”. Estenda o trabalho realizado com funcionalidades apropriadas que permitem a um automobilista ligar para uma linha de emergência (e.g. 112), dar a sua posição, e solicitar uma rota para evacuação; com o nome do da rua onde o automobilista está, o sistema retorna a rota de evacuação. Implemente esta funcionalidade, considerando tanto pesquisa exata, assim como pesquisa aproximada, das strings identificativas dos nomes das ruas fornecidas. Para pesquisa exata, caso o nome de rua não exista, deverá retornar mensagem de lugar desconhecido. Para a pesquisa aproximada, deverá retornar os nomes de ruas mais próximos, ordenados por similaridade.

Estas novas funcionalidades deverão ser integradas no trabalho já realizado para a primeira parte. Avalie a complexidade (teórica e empiricamente) dos algoritmos implementados em função dos dados de input usados.
\end{quote}



\subsubsection{Descrição da Implementação}

No seguimento das aulas de pesquisa aproximada e exata foram implementados algoritmos de pesquisa exata, algoritmos de distância de edição e os equivalentes algoritmos de pesquisa aproximada (\textit{fuzzy search}).

O utilizador do programa seleciona o algoritmo a utilizar
\end{document}