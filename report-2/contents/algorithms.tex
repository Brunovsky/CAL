\documentclass[relatorio.tex]{subfiles}
% Load Map

\newcolumntype{b}{>{\hsize=1.44\hsize}X}
\newcolumntype{n}{X}
\newcolumntype{s}{>{\hsize=.56\hsize}X}

\begin{document}
\section{Algoritmos}
\label{sec:alg}

Implementamos \emph{Boyer-Moore} (com \textit{shift rules} genéricas) e \emph{Knuth-Morris-Pratt} para pesquisa exata.

Para distância de edição estão implementadas as distâncias de \emph{Levenshtein} (inserção, remoção, substituição), \emph{Damerau-Levenshtein} simplificado (\emph{Optimal string alignment distance} segundo a \textit{Wikipédia}) e \emph{Damerau-Levenshtein} (inserção, remoção, substituição, transposição) -- para pesquisa aproximada estão os implementados os equivalentes a estes.

\paragraph{Implementação: Boyer-Moore}
As regras definidas pelo algoritmo (\emph{bad character rule} e \emph{good suffix rule}) estão feitas de forma genérica, permitindo várias estruturas internas.

A \emph{bad character rule} tem duas implementações disponibilizadas: uma com tabela de tamanho $L\times P$, outra com uma tabela mapeada de tamanho $E\times P$, requirindo menos memória.

A classe \class{boyer_moore} encapsula as duas regras para os algoritmos de pesquisa. As implementações do algoritmo e da regra \emph{good suffix rule} seguem as orientações do artigo original\cite{boyermoore}.

\paragraph{Implementação: Knuth-Morris-Pratt}
A implementação segue as orientações dos slides das teóricas. O preprocessador é, à semelhança do algoritmo anterior, uma classe à parte.\\

Em ambos os algoritmos anteriores os preprocessamentos de $P$ são reutilizáveis -- é feito uma vez e aplicado a todas as ruas a pesquisar.

\paragraph{Implementação: Levenshtein}
A implementação segue mais uma vez o exemplo das teóricas. Há duas implementações: uma usando uma matriz $P\times T$ e outra usando dois vetores de tamanho $T$, poupando memória.

\paragraph{Implementação: Damerau-Levenshtein}
As implementações seguem as orientações bastante detalhadas em \cite{damerau}.

O algoritmo simplificado tem uma implementação usando uma matriz $P\times T$ e outra implementação usando uma matriz $3\times T$. O algoritmo \emph{Damerau-Levenshtein} não pode ser simplificado de uma forma análoga aos anteriores e tem de ser implementado usando uma matriz $P\times T$.

\paragraph{Implementação: Pesquisa aproximada}
Os algoritmos de pesquisa aproximada são adaptados de forma direta dos algoritmos de distância de edição -- nestes casos, estamos interessados em saber a menor das distâncias das \textit{substrings} de $T$ a $P$, em vez da distância entre $T$ e $P$.
\end{document}