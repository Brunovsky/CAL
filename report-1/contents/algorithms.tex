\documentclass[relatorio.tex]{subfiles}
% Load Map

\newcolumntype{b}{>{\hsize=1.44\hsize}X}
\newcolumntype{n}{X}
\newcolumntype{s}{>{\hsize=.56\hsize}X}

\begin{document}
\section{Algoritmos}
\label{sec:alg}

Necessitamos de algoritmos que assegurem e realizem três tarefas distintas:

\begin{itemize}
\item \textbf{Deteção de conetividade rápida (em $\Theta_t(V)$)}\\
Determinar a conetividade do grafo em tempo linear.

\item \textbf{Determinação do caminho mais curto}\\
Encontrar o caminho mais curto entre quaisquer dois vértices.

\item \textbf{Determinação do caminho mais rápido (simulação)}\\
Usando o algoritmo de \alg{Dijkstra}, uma função heurística
de tempo, e uma dinâmica de grafo que gere -- automaticamente e
de forma aproximada -- o trânsito do grafo, é possível visualizar
a simulação de uma viagem de um utente através do grafo.
\end{itemize}

Para este fim implementámos as quatro classes de algoritmos na tabela \ref{tab:algorithms}.

\begin{table}[hbpt]
\caption{Algoritmos Implementados}
\label{tab:algorithms}
\begin{tabularx}{\textwidth}{rX}
\makecell[cr]{\textbf{Greedy BFS}\\Distância}&
Para encontrar uma aproximação do
(possivelmente o) caminho mais curto
entre dois vértices no grafo.\\

\makecell[cr]{\textbf{Dijkstra}\\Distância}&
Para encontrar o caminho mais curto
entre dois vértices do grafo.\\

\makecell[cr]{\textbf{A*}\\Distância}&
Para encontrar o menor caminho
entre dois vértices, usando a heurística
de distância ao destino
para melhorar a prioritização dos vértices.\\
\end{tabularx}
\end{table}



\FloatBarrier
\subsection{Conetividade}
\label{subsec:connectivity}

Para determinar a conetividade do grafo $G$ a partir de um vértice $v$,
ou seja, determinar a \textsl{closure} $G'_v$ de $G$ que inclui $v$, usamos
o algoritmo \alg{Breadth First Search}, sem destino. Ver Algoritmo~\ref{alg:bfs}.
Assumamos que $G'_v$ tem o conjunto de vértices $V'_v$ e de arestas $E'_v$.

\paragraph{Complexidade Temporal}
A complexidade temporal de \alg{Breadth First Search} (Pesquisa em largura)
é $\Theta(E + V)$. Prova. Todas as instruções do algoritmo têm complexidade temporal
constante ($\Theta(1)$). O \looop{while} exterior executa $V_v'$ vezes e o
\looop{for} interior executa $E'_v$ vezes no total.
Logo $\Theta_t=\Theta(E'_v+V'_v)=\Theta(E'_v)+\Theta(V'_v)=\Theta(E)+\Theta(V)=\Theta(E+V)$.

\paragraph{Complexidade Espacial}
A complexidade espacial de \alg{Breadth First Search} é $\Theta_e(V)$. Todas as
variáveis exceto \sf{queue} têm espaço de memória constante, e \sf{queue}
armazena no máximo $V'_v$ vértices. Logo $\Theta_e=\Theta(V'_v)=\Theta(V)$.

\paragraph{Correção}
A ideia básica subjacente a este algoritmo

As considerações acima também se aplicariam ao caso em que o algoritmo
terminasse cedo (\emph{early exit})
O algoritmo \alg{BFS} é utilizado como algoritmo auxiliar dos restantes,
para determinar quais são os vértices alcançáveis. Como alternativa
poderia-se usar o algoritmo \alg{Depth First Search}, com complexidade
temporal e espacial equivalentes.

\begin{algor}{Breadth First Search($v$ \var{origin})}{alg:bfs}
\State queue = new Queue() \Comment Simple Queue
\State queue.push(\var{origin})
\\
\While {not queue.empty()}
    \State current = queue.front()
    \State queue.pop()
    \For {next in adj(current)}
        \If {accident in next}
            \State continue
        \EndIf
        \If {not next.visited}
            \State next.visited = true
            \State queue.push(next)
        \EndIf
    \EndFor
\EndWhile
\end{algor}


\FloatBarrier
\subsection{Determinação de caminho mais curto}
\label{subsec:shortestpathalg}

Para determinar o caminho mais curto (em termos de distância euclidiana)
disponibilizamos três algoritmos distintos: \alg{Greedy Best-First Search}
-- pesquisa gananciosa muito rápida e não necessariamente correta --
\alg{Dijkstra} e \alg{A*}. Ver Algoritmos \ref{alg:gbfs}, \ref{alg:dijkstralateexit},
\ref{alg:dijkstraearlyexit}, \ref{alg:astar}.

Em todos os algoritmos, é assumido que uma limpeza do campo $v.\sfmath{cost}$
foi realizada pelo menos para todos os vértices de $V'_v$,
designando o valor inicial $v.\sfmath{cost}=\infty, \forall v\in V'_v$.

\paragraph{Complexidade Temporal}
A complexidade temporal dos algoritmos é $\Theta_t((V + E)\ln V)$ se \class{PriorityQueue}
for implementada como uma Árvore Binária. Prova.
O \looop{while} exterior executa no máximo $V'_v\leq V$ vezes e o \looop{for}
interior executa no máximo $E'_v\leq E$ vezes no total. As operações de \sf{push}
e \sf{pop} executam em tempo $\Theta(\ln \sizeof(\sfmath{pqueue}))$, e como o tamanho de
\sf{queue} é no máximo $V'_v\leq V$ a qualquer momento então $\Theta(\ln \sizeof(\sfmath{pqueue}))
=\Theta(\ln V)$. Logo $\Theta_t=\Theta(E+V)\cdot\Theta(\ln V)=\Theta((E+V)\ln V)$.

\paragraph{Complexidade Espacial}
A complexidade espacial dos algoritmos é $\Theta_e(V)$. A lógica é idêntica
ao algoritmo \alg{BFS} de conetividade (ver \ref{subsec:connectivity}).

\paragraph{Correção Dijkstra}
Inicialmente todos os vértices $v$ têm \sf{cost} infinito. Logo a primeira vez
que um vértice $\sfmath{next}\neq\var{origin}$ é encontrado no \looop{for} interior,
\sf{next.cost} é infinito e a desigualdade $\sfmath{newcost}<v.\sfmath{next.cost}$ é
automaticamente verdadeira, designando o valor \sf{newcost} a \sf{next.cost}.
Sabendo que $\var{origin}.cost$ é $0$ pode-se provar por indução sobre a iteração
$j$ do \looop{while} que $v.\sfmath{cost}$ é o caminho mais curto de \var{origin} a
$v$ usando apenas os vértices já iterados $\{v_1=\var{origin},v_2,v_3,\dots,v_j\}$.
Isto demonstra a correção do algoritmo \ref{alg:dijkstralateexit} (\emph{late exit}).
Suponhamos que o algoritmo \ref{alg:dijkstraearlyexit} é inválido, e aquando do retorno
na instrução \sf{break} o melhor caminho não é o determinado. Pela indução deduzida acima
sabemos que o melhor caminho determinado, digamos com comprimento $l$, usa apenas os vértices
$\{v_1=\var{origin},v_2,\dots,v_k=\var{destination}\}$. Portanto o melhor caminho
usa um vértice $v'$ que não está neste conjunto, e como tal ainda não foi iterado
no \looop{while}. Logo $v'.\sfmath{cost}>v_k.\sfmath{cost}=\sfmath{destination.cost}$.
Se $d$ é o comprimento do melhor caminho, então $v'.\sfmath{cost}<d$ pois as distâncias
são não negativas. Mas então $d>\sfmath{destination.cost}=l$. Contradição.

\begin{algor}{Greedy Best-First Search($v$ \var{origin}, $v$ \var{destination})}{alg:gbfs}
\State pqueue = PriorityQueue()
\State pqueue.push(\var{origin})

\While {not pqueue.empty()}
    \State current = pqueue.front()
    \State pqueue.pop()

    \For {next in adj(current)}
        \If {accident in next}
            \State continue \Comment Clear vertices only
        \EndIf
        \If {not next.visited}
            \State next.visited = true
            \State next.priority = distance(next, \var{destination}) \Comment Greedy
            \State pqueue.push(next)
        \EndIf
    \EndFor
\EndWhile
\end{algor}

\begin{algor}{Dijkstra($v$ \var{origin}) \Comment Late Exit}{alg:dijkstralateexit}
\State pqueue = PriorityQueue() \Comment Supporting decrease key
\State pqueue.push(\var{origin})

\While {not pqueue.empty()}
    \State current = pqueue.front()
    \State pqueue.pop()

    \For {next in adj(current)}
        \If {accident in next}
            \State continue \Comment Clear vertices only
        \EndIf
        \State newcost = current.cost + distance(current, next)
        \If {newcost < next.cost} \Comment Found a shorter path
            \State next.cost = newcost
            \State next.priority = newcost
            \State next.previous = current
            \State pqueue.push(next) \Comment Decrease key if already present
        \EndIf
    \EndFor
\EndWhile
\end{algor}

\begin{algor}{Dijkstra($v$ \var{origin}, $v$ \var{destination}) \Comment Early Exit}{alg:dijkstraearlyexit}
\State pqueue = PriorityQueue() \Comment Supporting decrease key
\State pqueue.push(\var{origin})

\While {not pqueue.empty()}
    \State current = pqueue.front()
    \State pqueue.pop()

    \If {current == \var{destination}}
        \State break
    \EndIf

    \For {next in adj(current)}
        \If {accident in next}
            \State continue \Comment Clear vertices only
        \EndIf
        \State newcost = current.cost + distance(current, next)
        \If {newcost < next.cost} \Comment Found a shorter path
            \State next.cost = newcost
            \State next.priority = newcost
            \State next.previous = current
            \State pqueue.push(next) \Comment Decrease key if already present
        \EndIf
    \EndFor
\EndWhile
\end{algor}

\begin{algor}{A*($v$ \var{origin}, $v$ \var{destination})}{alg:astar}
\State pqueue = PriorityQueue() \Comment Supporting decrease key
\State pqueue.push(\var{origin})

\While {not pqueue.empty()}
    \State current = pqueue.front()
    \State pqueue.pop()

    \If {current == \var{destination}}
        \State break
    \EndIf

    \For {next in adj(current)}
        \If {accident in next}
            \State continue \Comment Clear vertices only
        \EndIf
        \State newcost = current.cost + distance(current, next)
        \If {newcost < next.cost} \Comment Found a shorter path
            \State next.cost = newcost
            \State next.priority = newcost + distance(next, \var{destination}) \Comment A*
            \State next.previous = current
            \State pqueue.push(next) \Comment Decrease key if already present
        \EndIf
    \EndFor
\EndWhile
\end{algor}


\FloatBarrier
\subsection{Determinação do caminho mais rápido entre vértices}
\label{subsec:fastestpath}

Para determinar o caminho mais rápido usamos uma função heurística simples,
chamemos-lhe \function{time}, capaz de avaliar o tempo que um veículo demora a
atravessar uma aresta, baseado no comprimento da rua subjacente e no seu
congestionamento (número de carros presentes naquele momento, face ao máximo).

Aquando da execução deste algoritmo é tirada uma \emph{fotografia} do estado do grafo
-- do género de um satélite -- o que significa que a validade do caminho descoberto
está limitada à conformidade de $G$ com esta \emph{fotografia}.

Para simular a complexa dinâmica do trânsito no grafo $G$, este algoritmo é
executado múltiplas vezes seguidas, de forma interativa: o condutor percorre apenas
uma rua, ou até apenas uma aresta, e então o caminho óptimo é recalculado tendo em
consideração um novo estado do grafo $G$ -- uma nova \emph{fotografia} -- idem
até chegar ao seu destino. Ver Algoritmo \ref{alg:dijkstrasimulation}.

Visto que a única diferença deste algoritmo de \alg{Dijkstra} com o algoritmo
de distância é o peso, e a função heurística tem complexidade temporal constante,
a complexidade deste algoritmo é igual: $\Theta_t((E+V)\ln V)$ e $\Theta_e(V)$, tal
como a sua correção.

\begin{algor}{Dijkstra($v$ \var{origin}, $v$ \var{destination}) \Comment Simulation}{alg:dijkstrasimulation}
\State pqueue = PriorityQueue() \Comment Supporting decrease key
\State pqueue.push(\var{origin})

\While {not pqueue.empty()}
    \State current = pqueue.front()
    \State pqueue.pop()

    \If {current == \var{destination}}
        \State break
    \EndIf

    \For {next in adj(current)}
        \If {accident in next}
            \State continue \Comment Clear vertices only
        \EndIf
        \State newcost = current.cost + time(current, next) \Comment Different weight
        \If {newcost < next.cost} \Comment Found a faster path
            \State next.cost = newcost
            \State next.priority = newcost
            \State next.previous = current
            \State pqueue.push(next) \Comment Decrease key if already present
        \EndIf
    \EndFor
\EndWhile
\end{algor}
\end{document}