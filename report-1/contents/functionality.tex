\documentclass[relatorio.tex]{subfiles}

\begin{document}
\section{Funcionalidades Implementadas}
\label{sec:functionality}

Sobre o mapa é possível escolher um vértice ou uma aresta e causar ou reparar um acidente grave.
Para além disso, é possível gerir o trânsito das arestas.
Existe ainda uma opção que permite escolher o tipo de algoritmo a usar, de entre os
disponíveis, ou se pretende usar uma simulação que dinamiza andeterminação do caminho
mais rápido entre um vértice de origem e outro de destino, escolhidos pelo utente.
Todas as funcionalidades implementadas solicitam ao utente a introdução de dados passo a passo,
especificando o que é pretendido que seja introduzido e reagindo em conformidade
com o input do utente. Por conseguinte, é fornecido o resultado das suas
ações de forma gráfica e de fácil deteção e compreensão.

\begin{table}[hbpt]
\caption{Funcionalidades Implementadas}
\label{tab:functionalities}
\begin{itemize}
\setlength{\parskip}{.2\parskip}
\item Escolher o mapa no qual o utente pretende navegar sendo necessário introduzir o nome do mesmo;
\item Acidentar um vértice indicando o seu id, disponibilizado pela interface gráfica;
\item Acidentar uma aresta indicando o seu id, disponibilizado pela interface gráfica;
\item Repara um vértice indicando o seu id, disponibilizado pela interface gráfica;
\item Reparar uma aresta indicando o seu id, disponibilizado pela interface gráfica;
\item Editar a informação de uma aresta, sendo necessário indicar o seu id, e escolhendo uma das seguintes opções:
    \begin{itemize}
    \item Adicionar carros à aresta, indicando a quantidade sendo esta somada à atual;
    \item Remover carros à aresta, indicando a quantidade a ser retirada à atual;
    \end{itemize}
\item Encontrar o caminho mais rápido, entre dois vértices indicados pelo utente, usando uma das seguintes opções:
    \begin{itemize}
    \item Algoritmo \alg{Greedy Best-First Search};
    \item Algoritmo \alg{Dijkstra} aplicado a todos os vértices do grafo;
    \item Algoritmo \alg{Dijkstra} aplicado até ao vértice de destino;
    \item Algoritmo \alg{A*};
    \item Simulação, aplicando variações de trânsito a cada aresta;
    \item Simulação, aplicando variações de trânsito a cada estrada;
    \end{itemize}
\item Avaliação da performance dos algoritmos anteriores usando Benchmarking e sendo indicado o número de iterações a serem testar;
\item Verificar a informação relativa à interface gráfica.
\end{itemize}
\end{table}
\end{document}