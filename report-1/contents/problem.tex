\documentclass[relatorio.tex]{subfiles}

\begin{document}
\section[Interpretação do Problema]{Interpretação do Tema e do Problema}
\label{sec:problem}

\subsection{Identificação do Problema}
\label{subsec:identproblem}

Após uma a análise detalhada do tema anteriormente descrito,
tornou-se evidente o problema em questão, problema este coincidente
com o objetivo principal do trabalho, isto é, obter o melhor percurso
alternativo, tendo em conta a segurança e o tempo, para cada utente.
Deste modo procedeu-se então à formalização do problema decompondo
este em quatro subproblemas:

\begin{enumerate}[listparindent=2em]
\item {\bfseries Cálculo do melhor percurso para cada utente tem em conta apenas a distância.}

Numa primeira fase do problema o objetivo passa por simplesmente determinar o percurso mais curto entre dois pontos definidos pelo utente, isto tendo em conta o grafo considerado para tal efeito. Nesta fase ignora-se a maior parte dos fatores intervenientes no que é considerado o objetivo final deste trabalho, e, portanto, considera-se apenas a distância entre dois pontos.

Assim numa primeira fase, o problema resume-se a um problema trivial de caminho mais curto tendo em conta a distância, sendo este resolúvel, recorrendo a determinados algoritmos que se encontram descritos na secção referente à descrição da solução implementada.

\item {\bfseries Cálculo do melhor percurso para cada utente tem em conta a distância mais os locais acidentados.}

Numa segunda fase, na qual já existe um ou vários algoritmos para resolver o problema base do tema deste trabalho, é introduzido um novo fator no mesmo, fator este referente à introdução de locais acidentados na rede de estradas considerada. 

Deste modo passam, portanto, a existir locais a ser evitados e por sua vez inacessíveis devido ao seu estado, no entanto, a introdução deste novo fator, não altera o funcionamento dos algoritmos isto porque, os locais acidentados passam a ser tidos em conta como locais inexistentes e consequentemente passam a ser ignorados pelo algoritmo no cálculo do melhor percurso para cada utente. 

\item {\bfseries Cálculo do melhor percurso para cada utente tem em conta a distância, os locais acidentados, e o tempo.}

Numa terceira fase, é introduzido o fator tempo que passa a ser considerado pelos algoritmos como o peso da aresta no grafo. Este tempo é obtido tendo em conta a distância de cada estrada, já utilizada nos subproblemas anteriores, e ainda a sua velocidade máxima. 

Este cálculo consiste numa operação matemática simples, que traduz na divisão da distância pela velocidade, obtendo-se assim o novo fator, isto é, o tempo que se demora a percorrer uma determinada estrada.

\item {\bfseries Cálculo do melhor percurso para cada utente tem em conta a distância, os locais acidentados, o tempo e a ocupação da rede de estradas considerada.}

Numa quarta e última fase introduz-se por fim o último fator que é a ocupação da estrada, isto é, o número de carros presentes num troço tendo em conta a capacidade máxima para esse mesmo troço.

Este novo fator é introduzido no trabalho na medida em que a velocidade com que uma determinada estrada é percorrida, tem em conta a sua ocupação. 

Para tal é tido em conta a relação linear velocidade-densidade na qual um aumento de densidade provoca uma diminuição da velocidade na estrada. Deste modo, quando o utente pretende verificar qual o melhor percurso para ir de um ponto A a um ponto B, o algoritmo fornece os resultados tendo em conta o estado do grafo, isto é, a ocupação da rede de estradas nesse momento, pois esta varia dinamicamente. Quando densidade de carros é máxima, ou seja, quando a ocupação atinge o limite máximo a velocidade é praticamente nula pois a estrada encontra-se congestionada.
\end{enumerate}



\subsection{Formalização do problema}
\label{subsec:formal}

\subsubsection{Dados de entrada}
\label{subsubsec:formalin}

Os dados de entrada neste trabalho correspondem à inicialização
do grafo que posteriormente será utilizado para visualização dos
dados de saída e sobre o qual irão operar os algoritmos de caminho
mais curto descritos na secção referente à descrição da solução implementada.
Para tal é necessário fazer um pré-processamento dos dados de entrada para
que estes algoritmos possam ser devidamente executados. 

Este pré-processamento encontra-se definido e explicado na secção \ref{sec:loadmap}.

\subsubsection{Dados de saída}
\label{subsubsec:formalout}

Os dados de saída neste trabalho correspondem
ao melhor percurso possível a ser percorrido pelo utente no instante
em que este pretende se deslocar entre dois pontos, evitando todos
os locais acidentados pelo caminho.

Estes dados podem ser obtidos visualmente através da GUI
e à qual se encontram associadas um conjunto de cores que
descrevem o funcionamento do mesmo.

\begin{table}
\caption{Cores da Interface}
\label{tab:colors}
\begin{itemize}
\item Azul -- corresponde a um vértice no seu estado normal.
\item Vermelho -- corresponde a vértices ou arestas acidentados, na medida em que estes
não podem ser utilizados por qualquer algoritmo na pesquisa do melhor percurso
possível entre dois pontos.
\item Verde -- corresponde ao melhor percurso possível entre dois pontos,
tendo em conta a situação atual do grafo.
\item Ciano -- corresponde ao melhor percurso possível previsto entre dois pontos.
Esta cor encontra-se associada aos algoritmos que envolvem simulação pois
implicam dinamismo no percorrer deste percurso que pode ou não ser variável
dependendo dos acontecimentos a ele associados.
\item Magenta -- corresponde aos vértices selecionados pelo utente,
tanto o vértice de origem como o de chegada.
\item Amarelo -- corresponde à cor dos vértices não alcançáveis.
A cada execução de um algoritmo para encontrar o melhor percurso possível
é realizada uma pesquisa prévia que determina os vértices não alcançáveis.
\end{itemize}
\end{table}
\end{document}