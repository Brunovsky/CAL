\documentclass[relatorio.tex]{subfiles}
% Load Map

\begin{document}
\section{Inicialização de um Mapa}
\label{sec:loadmap}

\subsection{Rotina de criação de ficheiros de dados}
\label{subsec:routinefiles}

Vamos agora descrever, muito sucintamente, o
procedimento para se extrair um novo mapa de
\url{https://www.openstreetmap.org/}
e incluí-lo no nosso projeto. O processo não
deverá demorar mais do que dois minutos.

No site \osm{}, depois de delimitada a região, extraímos o
ficheiro-mapa \file{map.osm} usando (preferencialmente)
a Overpass API. Depois corremos o ficheiro na aplicação \exe{OpenStreetMapParser}, com o ficheiro exportado.

O \exe{Parser} gera três ficheiros: um contendo a informação
relativa a $V$, o ficheiro \file{nodes}; outro
contendo informação relativa a $R$, o ficheiro \file{roads}
-- que descreve a topologia de $G$; e outro contendo a
informação de $E$, descrevendo a decomposição das ruas
-- o ficheiro \file{subroads}.

Além destes três ficheiros, um quarto ficheiro de configuração
\file{meta} deve ser manualmente criado com a informação geral
do mapa, e algumas opções-extra para customizar a inicialização
de~$G$.

Este ficheiro de configuração suporta os campos presentes na
tabela \ref{tab:meta}, em estilo \sf{key=val;}

\begin{table}[hbpt]
\caption{Opções do ficheiro \file{meta}}
\label{tab:meta}
\begin{tabularx}{\textwidth}{rX}
&Obrigatórios:\\
\textbf{min\_longitude} &A longitude mínima de $G$.\\
\textbf{max\_longitude} &A longitude máxima de $G$.\\
\textbf{min\_latitude} &A latitude mínima de $G$.\\
\textbf{max\_latitude} &A latitude máxima de $G$.\\
\textbf{nodes} &O número de nodes (de $M$ e de $G$).\\
\textbf{edges} &O número de edges (de $M$ e de $G$).\\
\\
&Opcionais:\\
\textbf{density} &(Opcional) A densidade (visual) dos vértices.\\
\textbf{width} &(Opcional) A largura preferida de $G$.\\
\textbf{height} &(Opcional) A altura preferida de $G$.\\
\textbf{boundaries} &(Opcional) Delimitar a área do grafo
com vértices-fantasma.\\
\textbf{oneway} &(Opcional) Sobrepõr o parâmetro direcional no ficheiro
\file{roads} e definir todas as ruas como unidirecionais.\\
\textbf{bothways} &(Opcional) Sobrepõr o parâmetro direcional no ficheiro
\file{roads} e definir todas as ruas como bidirecionais.\\
\textbf{straight\_edges} &(Opcional) Desenhar todas as arestas retílineas.\\
\end{tabularx}
\end{table}

Os limites geográficos, o número de vértices e o número de arestas
são todos fornecidos pelo \exe{Parser} e definem a geometria de $M$.
O ficheiro \file{meta} é lido com expressões regulares para cada
parâmetro, assumindo a sintaxe \sf{key=val;} portanto as opções
podem estar distribuídas de qualquer forma.

Os quatro ficheiros deverem ser colocados na pasta \folder{resources}.
A convenção de nomenclatura dos ficheiros é simples: escolhido
um nome representativo qualquer, por exemplo \file{city},
os quatro ficheiros deverão ser chamados \file{city_meta.txt},
\file{city_nodes.txt}, \file{city_roads.txt} e \file{city_subroads.txt}.



\subsection{Cálculo das dimensões do grafo}

Os vértices listados no ficheiro \file{nodes} têm a sua localização
apresentada em coordenadas geográficas. Para representar os vértices
$v$ em \class{GraphViewer} estas coordenadas têm de ser
convertidas em coordenadas cartesianas $(x,y)$, de modo que
se mantenha a proporcionalidade do mapa.

As dimensões de $M$ são calculadas diretamente através das fórmulas
$\sfmath{height (km)}=110.574\cdot\delta\sfmath{latitude}$ e
$\sfmath{width (km)}=111.320\cdot\delta\sfmath{longitude}\cdot
|\cos(\overline{\sfmath{latitude}})|$.

A proporção \sf{height:width} de $M$ é também a proporção visual
$G$. Especificada uma largura (\textit{width}), altura (\textit{height})
ou densidade de vértices (\textit{density}) preferida no ficheiro
\file{meta}, determina-se as dimensões de $G$.

Para determinar as dimensões de $G$ a partir de uma densidade $d$,
note-se que $\sfmath{width}\times\sfmath{height}\times d=V$, e
portanto

\begin{equation*}
\sfmath{width}=\sqrt{\frac{V}{d\times\sfmath{height:width}}}
\end{equation*}

Por omissão usamos densidade $0.0001$.



\subsection{Ponto de partida e leitura dos ficheiros de dados}
\label{subsec:startmap}

Todos os ficheiros de dados são lidos usando expressões regulares
\sf{std::regex} da \sf{STL}. Isto permite que erros nos ficheiros
de dados sejam detetados de forma simples e direta, mas também
significa que qualquer mudança manual nestes deva ser feita com
cuidado. A ter em consideração as seguintes situações, que são
automaticamente detetadas:

\begin{itemize}
\item (Erro) Os parâmetros obrigatórios do ficheiro \file{meta} têm
de estar todos presentes.
\item (Warning) O número de vértices indicado em \file{meta} deve ser
igual ao número de nodes (linhas) lidos em \file{nodes}.
\item (Warning) O número de arestas indicado em \file{meta} deve ser
igual ao número de subroads (linhas) lidas em \file{subroads}.
\item (Erro) As latitudes e as longitudes lidas têm de estar dentro
dos limites geográficos especificados.
\end{itemize}

Em suma, para carregar um mapa seguem-se os passos:

\begin{enumerate}
\item  Lê-se o ficheiro de configuração \file{meta}.
\item Inicializa-se $G$ (e \class{GraphViewer})
com as dimensões calculadas.
\item Lê-se o ficheiro \file{nodes} e cria-se $V$.
\item Lê-se o ficheiro \file{roads} e cria-se $R$.
\item Lê-se o ficheiro \file{subroads} e cria-se $E$.
\end{enumerate}
\end{document}