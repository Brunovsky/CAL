\documentclass[relatorio.tex]{subfiles}

\newcommand*{\rarrow}[1]{\ensuremath{\overrightarrow{#1}}}
\newcommand*{\id}{\sf{id} }
\newcommand*{\pre}[1]{}

\newcolumntype{b}{>{\hsize=1.44\hsize}X}
\newcolumntype{n}{X}
\newcolumntype{s}{>{\hsize=.56\hsize}X}

\begin{document}
\section{Introdução}
\label{sec:intro}

\subsection{Definições preliminares}
\label{subsec:preliminaries}

Ao longo do relatório usamos as seguintes definições e conceitos:

\begin{definitions}
\item $G$ refere-se à abstração de grafo (ou à classe \class{Graph});

\item $M$ refere-se ao mapa \textsl{real} extraído de \osm{} e
representado por~$G$.

\item $V$ é o conjunto dos vértices de~$G$, ou a
sua cardinalidade num contexto numérico;

\item $E$ é o conjunto das arestas de~$G$, ou a sua
cardinalidade num contexto numérico;

\item $R$ é o conjunto das ruas, com nome ou não, de~$M$;

\item $v\in V$, $v_i\in V$ são vértices de $V$.

\item $e\in E$, $e_i\in E$ são arestas de $E$.
A aresta $e_{ij}$ é aquela que liga $v_i$ a $v_j$.

\item $r\in R$, $r_i\in R$ são ruas de $R$.

\item \file{meta}, \file{nodes}, \file{roads}, \file{subroads}
são ficheiros de dados que servem como \textsl{input} do programa.
Mais informação na subsecção \ref{subsec:routinefiles}.

\item $\Theta(\sfmath{expression}(n))$ refere-se à complexidade
temporal ou espcial de uma função ou algoritmo (\bigO).\\
Mais precisamente, $\Theta_t$ é a complexidade temporal,
e $\Theta_e$ a complexidade espacial de um algoritmo ou função.
\end{definitions}



\subsection{Tema 3}
\label{subsec:tema}

\subsubsection{Especificação do Tema}
\label{subsubsec:spec}

A especificação e descrição do Tema 3 -- Sistema de Evacuação
é a seguinte:

\begin{quote}
Um sistema de evacuação permite retirar em segurança os
utentes presentes numa área acidentada.
Considere a rede de auto-estradas de Portugal e suponha a
ocorrência de um acidente grave (incêndio, por ex) que obriga
ao corte de um troço da auto-estrada e desvio dos automóveis
em circulação.
Neste trabalho, pretende-se implementar um sistema que determine
os percursos alternativos que cada automóvel deve realizar.
Considere a existência de múltiplos automóveis nos vários troços
das auto-estradas, cada automóvel possui o seu destino. Cada
troço da auto-estrada possui uma capacidade limitada, isto é,
número máximo de automóveis que suporta (este valor não deve
ser ultrapassado, correndo o risco de parar o trânsito automóvel
nesse troço).  
O sistema deve indicar, para cada automóvel, o melhor percurso
alternativo que deve realizar para evitar o troço acidentado e
chegar ao seu destino em segurança, no menor tempo possível.

Avalie a conectividade do grafo, a fim de evitar que locais de destino se encontrem em zonas inacessíveis a partir do ponto onde se encontra o automobilista quando da ocorrência do acidente.
\end{quote}



\subsubsection{Descrição do Tema}

O tema do presente trabalho consiste num sistema de evacuação
que permite não só retirar em segurança os utentes presentes
numa área acidentada como também definir trajetos para os quais
serão evitadas zonas acidentadas. Para tal serão consideradas
diferentes redes de estradas nas quais ocorrerão acidentes graves
que por sua vez implicam o corte das mesmas.

Cada estrada possui uma capacidade limitada, isto é, possui um
número máximo de automóveis que suporta, existindo, portanto, o
risco de congestionamento do trânsito nos diversos troços existentes.

Assim torna-se evidente a necessidade de determinar percursos
alternativos para os utentes de modo a evitarem estas zonas
acidentadas, tendo este trabalho como principal objetivo, encontrar
o melhor percurso alternativo que cada utente deve realizar para
evitar o troço acidentado e chegar ao seu destino em segurança,
no menor tempo possível.
\end{document}