\documentclass[relatorio.tex]{subfiles}

\begin{document}
\section{Classes e Estruturas de Dados}
\label{sec:classdata}

\subsection{Filosofia da Implementação}
\label{subsec:philosophy}

A base do programa é a abstração \class{Graph}, baseada naquela
que é usada nos exercícios da aula prática $5$. É extendida
consideravelmente, suportando os algoritmos que listamos na Tabela
\ref{tab:algorithms} e a acidentação de vértices e de arestas,
representando acidentes nas vias e impossibilitando movimento.

No projeto usamos a versão \sf{1.2} da \sf{API GraphViewer} fornecida
pelos professores. Esta API é parcialmente encapsulada pela
classe \class{Graph}.

O ponto de partida normal do programa é a leitura de um conjunto
de ficheiros de texto -- já listados na subsecção
\ref{subsec:preliminaries} -- que descrevem um mapa extraído de
\osm{} e processado pelo \sf{OpenStreetMapsParser} (fornecido pelos
professores). Não existe outro ponto de partida para o programa.

O grafo $G$ procura representar com precisão e proporção um mapa
real. Como tal, $G$ é estático depois de inicializado, sendo que
certas funcionalidades do género \textsl{create} e \textsl{remove}
não estão disponíveis.


\subsection{Classes}
\label{subsec:graph}

\begin{table}[hbpt]
\caption{Classes Principais}
\label{tab:classes}
\begin{tabularx}{\textwidth}{lX}
\textbf{Graph}& A abstração de $G$ e a base do programa.
Corresponde na realidade ao sistema de ruas e interseções do mapa.
É estritamente orientado, (normalmente) planar e
portanto esparso em termos de arestas
(o que significa $V\sim E$, i.e. $\Theta(V)=\Theta(E)$).\\

\textbf{Vertex}& A abstração de um vértice~$v$.
Corresponde na realidade a uma interseção de ruas, ou a
uma parte de uma rua para delinear o seu trajeto curvilíneo.
Cada entrada do ficheiro \file{nodes} é um vértice~$v$.\\

\textbf{Road}& A abstração de uma rua~$r$ do mapa real.
Corresponde a uma rua no mapa real, e é constituído
por uma sequência (caminho) de arestas~$e_j$, $j=1,\dots,k$.
Cada entrada do ficheiro \file{roads} é uma rua~$r$.\\

\textbf{Edge}& A abstração de uma aresta~$e$.
Corresponde a uma porção maioritariamente retilínea
de um rua no mapa real, sem interseções ou bifurcações.
Cada entrada do ficheiro \file{subroads} é uma aresta~$e$.\\

\textbf{Subroad}& A abstração de uma porção de uma estrada~$s$.
À semelhança da abstração \class{Edge}, corresponde a uma porção
maioritariamente retilínea de uma rua no mapa real, sem interseções ou
bifurcações, embora esta seja destinada a gerir e armazenar informação
usada pela Edge. Cada entrada do ficheiro subroads é uma porção~$s$.
\end{tabularx}
\end{table}


\FloatBarrier
\subsubsection{Graph}
\label{subsubsec:graph}

\class{Graph} encapsula $V$, a \sf{API} de visualização do grafo e
ainda todos os algoritmos implementados, além das estruturas de dados
auxiliares à programação dinâmica destes .
A classe \class{Graph} contém os atributos:

\begin{itemize}
\item width, height: dimensões do mapa do grafo.
\item vertexSet: vetor de apontador para os vértices que
constituem o grafo.
\item accidentedVertexSet: vetor de apontadores para os vértices que
constituem o grafo e que se encontram intransitáveis. 
\item gv: apontador para o objeto da classe GraphViewer responsável pela representação visual do grafo.
\item scale: fator de escala do mapa visualizado no ecrã.
\end{itemize}


\subsubsection{Vertex}
\label{subsubsec:vertex}

Cada \class{Vertex} detém uma posição no mapa do
grafo (coordenadas cartesianas~$(x_i,y_i)$ inteiras)
projetada pela sua posição em $M$. Pode ou não estar
\acc{acidentado}. Cada \class{Vertex} é responsável por
gerir as arestas que dele saem -- o que efetivamente
significa que $V$ encapsula $E$.

A classe \class{Vertex} contém os seguintes atributos:

\begin{itemize}
\item id: inteiro positivo que identifica, de forma única, um vértice. 
\item x, y: coordenadas cartesianas. 
\item accidented: booleano que indica se o vértice se encontra
acidentado ou não. Os vértices acidentados estão cortados ao
transito impossibilitando a passagem de veículos. 
\item adj: vetor de arestas com origem neste vértice e que o ligam a outros distintos.
\item accidentedAdj: vetor de arestas com origem neste vértice que se estão atualmente intransitáveis.
\item graph: apontador para o objeto da classe Graph que contém o vértice e que o representa.
\end{itemize}


\subsubsection{Edge}
\label{subsubsec:edge}


Cada Edge encapsula, essencialmente, a informação da subroad que
é representada por uma aresta e os vértices que a delimitam.
A classe Edge contém os seguintes atributos:

\begin{itemize}
\item id: inteiro positivo que identifica uma aresta de forma única.
\item source: vértice de início da aresta.
\item dest: vértice de destino de aresta.
\item accidented: booleano que indica se a aresta se encontra intransitável ou não.
\item graph: apontador para o objeto da classe Graph que contém a aresta e que a representa.
\item subroad: apontador para o objeto da classe Subroad que esta aresta representa.
\end{itemize}


\subsubsection{Road}
\label{subsubsec:road}

Cada Road encapsula a informação que descreve a rua real em M: o seu nome e o comprimento.
	A classe Road contém os atributos seguintes:

\begin{itemize}
\item id: inteiro positivo que identifica uma estrada de forma única.
\item name: nome da estrada.
\item bothways: booleano que indica se a estrada é de dois sentidos ou não.
\item totalDistance: distância total da estrada.
\item maxSpeed: limite de velocidade da estrada calculado a partir da distância.
\end{itemize}


\subsubsection{Subroad}
\label{subsubsec:subroad}

Cada SubRoad contém a informação do mundo real sobre uma Edge.
A classe Subroad contem os seguintes atributos:

\begin{itemize}
\item distance: comprimento da sub estrada.
\item road: apontador para a rua à qual pertence.
\item actualCapacity: quantidade de carros presentes na sub estrada.
\item maxCapacity: quantidade máxima de carros que podem existir na sub estrada.
Para além dos atributos referidos, foram também usados
outros adicionais, nas classes Vertex e Edge, específicos
para certos atributos, que serão abordados mais á frente.
\end{itemize}


\end{document}